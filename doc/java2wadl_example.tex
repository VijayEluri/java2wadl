In this example, a WADL description is generated for generic JAX-RS items resource.
\\ \\
The \emph{Item}-class is a container for \textless name,description\textgreater\ elements. For simplicity's sake the generated XML Schema for this class is not included.
\lstset{basicstyle={\footnotesize \ttfamily}, language=Java, numbers=left, frame=none, tabsize=2, caption=Item Model label=Item.java}
\begin{lstlisting}
package at.ac.tuwien.infosys.java2wadl.items;

public class Item {
	public final String name = "";
	public final String description = "";
	
	public Item(String name, String description) {
		this.name = name;
		this.description = description;
	}
}
\end{lstlisting}
The \emph{ItemsResource}-class is the JAX-RS resource that will be converted into a WADL description.

\lstset{basicstyle={\footnotesize \ttfamily}, language=Java, numbers=left, frame=none, tabsize=2, caption=Items Resource label=Item.java}
\begin{lstlisting}
package at.ac.tuwien.infosys.java2wadl.items;

import java.util.*;

import javax.ws.rs.*;

@Path("/items")
public class ItemsResource {
	private Map<String, Item> items;

	@GET 
	@Path("/ids/{itemsid}")
	@Produces( { "application/json", "application/xml" })
	public Item getItem(@PathParam("itemsid") String itemId) {
		return getItems().get(itemId);
	}

	@PUT 
	@Path("/ids/{itemsid}")
	@Consumes( { "application/json", "application/xml" })
	public void putItem(@PathParam("itemsid") String itemId, 
			Item item) {
		getItems().put(itemId, item);
	}

	@DELETE 
	@Path("/ids/{itemsid}")
	public void deleteItem(@PathParam("itemsid") 
			String itemId) {
		getItems().remove(itemId);
	}

	public Map<String, Item> getItems() {
		if (null == items) {
			items = new HashMap<String, Item>();
			items.put("I01", new Item("I01", "Item1."));
			items.put("I02", new Item("I02", "Item2."));
		}
		return items;
	}
}
\end{lstlisting}

The resulting file contains a correct and complete mapping of the Java class to the WADL description.
\lstset{basicstyle={\footnotesize \ttfamily}, language=XML, frame=none, tabsize=2, caption=Items Resource label=Item.java}
\begin{lstlisting}
<?xml version="1.0"?>
<application xmlns="http://research.sun.com/wadl/2006/10"
xmlns:xs="http://www.w3.org/2001/XMLSchema">
  <grammars>
    <include href="at.ac.tuwien.infosys.java2wadl.items.Item.xsd">
    </include>
  </grammars>

  <resources base="/base">
    <resource path="items">
      <resource path="ids">
        <resource path="{itemsid}">
          <method href="#ItemsResource_putItem_String_Item" />

          <method href="#ItemsResource_deleteItem_String" />

          <method href="#ItemsResource_getItem_String" />
        </resource>
      </resource>
    </resource>
  </resources>

  <method name="PUT" id="ItemsResource_putItem_String_Item">
    <request>
      <representation mediaType="application/json"
      element="j2wns:item">
      </representation>

      <representation mediaType="application/xml"
      element="j2wns:item">
      </representation>
    </request>

    <response>
      <representation status="204">
      </representation>
    </response>
  </method>

  <method name="DELETE" id="ItemsResource_deleteItem_String">
    <response>
      <representation status="204">
      </representation>
    </response>
  </method>

  <method name="GET" id="ItemsResource_getItem_String">
    <response>
      <representation mediaType="application/json"
      element="j2wns:item" status="200">
      </representation>

      <representation mediaType="application/xml"
      element="j2wns:item" status="200">
      </representation>
    </response>
  </method>
</application>
\end{lstlisting}
