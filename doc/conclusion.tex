The evaluation of REST/WADL and SOAP/WSDL shows that Service-Oriented Architectures (SOA) and Web services in particular can be implemented in different ways. 
\\ \\
While REST/WADL is based on the paradigm that applications should publish their data on the Web through URIs, SOAP/WSDL sees the Web only as a means of transport for messages, which has a large impact on resulting software architectures. RESTful Web applications embrace the HTTP standard and apply its key abstractions in a direct manner (e.g. HTTPs methods, encryption via HTTPs). SOAP on the other hand can be seen as a layer of abstraction built on top of transfer protocols.
\\ \\
This article also shows that due to SOAPs extensible nature, the WS-* stack can become overly complex for applications. On the other hand a large library of re-usable software components such as transactions and security-helpers are available for SOAP. The RESTful architectural style does not provide any means for extensions and thus forces applications and frameworks to re-invent the wheel in many cases.
\\ \\
The comparison of WADL and WSDL shows that both standards can be used to describe Web services. While WADL is limited to HTTP only, WSDL (2.0) provides a transport-protocol-agnostic alternative which can be used to describe SOAP and also generic Web applications. 
\\ \\
The question remains, whether RESTful Web services need a description language at all. Proponents argue that description languages are necessary, because they improve machine-to-machine interaction and provide protocol agreements. Opponents criticize that any means of describing URIs of RESTful resources statically conflict with the key constraints of the paradigm.