This example demonstrates the generation of Java classes based on WADL descriptions.
\\ \\
\emph{Listing 7} shows the WADL-file that will be converted into a set of Java classes:
\lstset{basicstyle={\footnotesize \ttfamily}, language=XML, frame=none, tabsize=2, caption=Customer Service WADL description label=CustomerService.wadl}
\begin{lstlisting}
<?xml version="1.0"?>
<application xmlns="http://research.sun.com/wadl/2006/10"
	xmlns:xs="http://www.w3.org/2001/XMLSchema">
  <grammars>
    <include 
    	href="at.ac.tuwien.infosys.java2wadl.customerservice.Customers.xsd">
    </include>

    <include
    	href="at.ac.tuwien.infosys.java2wadl.customerservice.Customer.xsd">
    </include>

    <include
	    href="at.ac.tuwien.infosys.java2wadl.customerservice.Product.xsd">
    </include>
  </grammars>

  <resources base="/base">
    <resource path="customerservice">
      <method href="#CustomerService_getCustomers_" />

      <resource path="customers">
        <method href="#CustomerService_addCustomer_Customer" />

        <resource path="{id}">
          <method href="#CustomerService_getCustomer_String" />

          <method href="#CustomerService_updateCustomer_String_Customer" />

          <method href="#CustomerService_deleteCustomer_String" />
        </resource>
      </resource>

      <resource path="orders">
        <resource path="{orderId}">
          <resource path="products">
            <resource path="{productId}">
              <method href="#Order_getProduct_Long" />
            </resource>
          </resource>
        </resource>
      </resource>
    </resource>
  </resources>

  <method name="GET" id="CustomerService_getCustomers_">
    <response>
      <representation element="j2wns:customers" status="200">
      </representation>
    </response>
  </method>

  <method name="GET" id="CustomerService_getCustomer_String">
    <response>
      <representation mediaType="application/xml"
      element="j2wns:customer" status="200">
      </representation>
    </response>
  </method>

  <method name="PUT" id="CustomerService_updateCustomer_String_Customer">
    <request>
      <representation mediaType="application/xml"
      element="j2wns:customer">
      </representation>
    </request>

    <response>
      <representation status="200">
      </representation>
    </response>
  </method>

  <method name="POST" id="CustomerService_addCustomer_Customer">
    <request>
      <representation element="j2wns:customer">
      </representation>
    </request>

    <response>
      <representation status="200">
      </representation>

      <fault status="400">
      </fault>
    </response>
  </method>

  <method name="DELETE" id="CustomerService_deleteCustomer_String">
    <response>
      <representation status="200">
      </representation>
    </response>
  </method>

  <method name="GET" id="Order_getProduct_Long">
    <response>
      <representation element="j2wns:product" status="200">
      </representation>
    </response>
  </method>
</application>
\end{lstlisting}
The resulting Java classes:
\lstset{basicstyle={\footnotesize \ttfamily}, language=Java, numbers=left, frame=none, tabsize=2, caption=Generated Customer Service Java Class, label=CustomerService.java}
\begin{lstlisting}
package test;

import javax.ws.rs.*;

@Path("/base/customerservice")
public class CustomerService {

	@GET
	public Customers getCustomers() {
		// TODO: Auto-generated method stub
		return null;
	}

	@POST
	@Path("/customers")
	public Response addCustomer(Customer customer) 
			throws WebApplicationException {
		// TODO: Auto-generated method stub
		return null;
	}

	@GET
	@Path("/customers/{id}")
	@Produces("application/xml")
	public Customer getCustomerXML(@PathParam("id") String id) {
		// TODO: Auto-generated method stub
		return null;
	}

	@PUT
	@Path("/customers/{id}")
	@Consumes("application/xml")
	public Response updateCustomerXML(@PathParam("id") String id, 
		Customer customer) {
		// TODO: Auto-generated method stub
		return null;
	}

	@DELETE
	@Path("/customers/{id}")
	public Response deleteCustomer(@PathParam("id") String id) {
		// TODO: Auto-generated method stub
		return null;
	}

	@GET
	@Path("/orders/{orderId}/products/{productId}")
	public Product getProduct(@PathParam("orderId") String orderId, 
			@PathParam("productId") String productId) {
		// TODO: Auto-generated method stub
		return null;
	}
}
\end{lstlisting}
Depending data models are also generated, but left out for the sake of simplicity.