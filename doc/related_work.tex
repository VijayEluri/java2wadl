\subsection{REST/WADL vs. SOAP/WSDL}
The subject of SOAP vs. REST has been an ongoing discussion over the last couple of years, while the field of description languages for Web applications has only recently gained popularity.
\\ \\
In case of SOAP vs. REST there are a number of papers that deal with the subject. It is also often discussed in the blogosphere. For example \cite{BIGREST} gives a detailed comparison of REST and SOAP, but does not cover the subject of WADL vs. WSDL (most likely because WADL was relatively new at the time of writing of this article). The article compares the two technologies on several levels: it analyzes the \emph{architectural principles}, \emph{conceptual and technological decisions} and \emph{concrete implementations}.
\\ \\
Another article that deals with the subject of SOAP vs. REST but also provides details for WADL vs. WSDL is \cite{WADLWSDL}. Similar to \cite{BIGREST} the authors introduce the underlying technologies and then compare SOAP/REST and WADL/WSDL in a general way without going into too much detail. In case of WADL vs. WSDL, they investigate \emph{SOAP interfaces} vs. \emph{REST resources}, \emph{transport protocols}, \emph{message exchange patterns} and the support of (server-side) \emph{state}.
\\ \\
\cite{BRINGBACKWEB} only analyzes the differences between SOAP and REST, but manages to give a complete overview of the key characteristics of each technology and is able to show \emph{conceptual} and \emph{technological} differences.
\\ \\
Chapter 10 of \cite{RESTFULWEB} offers a critical comparison of RESTful Web services and Big Web services. The authors criticize that the WS-* stack is overly complex and that it is not compatible with the originally intent "resource-oriented" nature of the Web. They also show RESTful alternatives to some WS-* extensions such as transactions.

\subsection{Java2Wadl2Java}

As of May 2008 there are two major contenders for the Java2Wadl2Java toolkit: 
\begin{itemize}
\item{The {\bf Jersey\footnote{\url{https://jersey.dev.java.net/}}} project is an open source community that provides an implementation for the \emph{JSR-311: JAX-RS - Java API for RESTful Web Services}. The main goal of this project is to provide a complete toolkit to develop RESTful Web applications. Thus it also offers means to generate WADL descriptions for RESTful resources.}
\item{{\bf wadl2java\footnote{\url{https://wadl.dev.java.net/}}} is a WADL compiler that allows to generate Java sources for provided WADL descriptions.}
\end{itemize}
