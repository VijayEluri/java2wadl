\subsubsection{API/Implementation}
The Java2Wadl2Java library is designed for the JVM and implemented in the Java Programming Language. It was developed specifically for J2SE 6.0.
\\ \\
The library is separated into an API and a default implementation. All implementation is written against the API, which makes it easy to exchange critical parts of the system at a later time. More important, however, is the fact that all type information which is available externally has to be part of the API. No implementation-specific information is allowed to leak. Separation of the API from the implementation ensures desirable software properties such as \emph{data hiding} and \emph{programming against interfaces}.
\\ \\
API-specific files can be found in \emph{src/main/java/api} and are typically Java Interfaces. The default implementation files are located in \emph{src/main/java/impl}. Unit tests are stored in \emph{src/test/java}.

\subsubsection{Entry points}
Java2Wadl2Java provides two main Interfaces to access the toolkit: \emph{IJava2Wadl} which is used to generate WADL files from Java classes and \emph{IWadl2Java} which works the other way around. An additional interface called \emph{IApplicationParser} is available which can be used to map WADL-XML-files onto an \emph{IAppliation} data-structure. 
\\ \\
\emph{IJava2Wadl} can be accessed with the \emph{generate} method which creates a helper data-structure containing the resulting WADL description and XML schemas. The method can also be called with an optional \emph{base URI} that is used as an enclosing namespace for all included resources.
\lstset{basicstyle={\footnotesize \ttfamily}, language=Java, numbers=left, frame=none, tabsize=2, caption=The IJava2Wadl Interface, label=IJava2Wadl}
\begin{lstlisting}
package at.ac.tuwien.infosys.java2wadl;

public interface IJava2Wadl {
	Wadl generate(Class<?> type) throws WadlException;

	Wadl generate(Class<?> type, String baseUri) throws WadlException;
}
\end{lstlisting}
\emph{IWadl2Java} can be accessed in a similar manner: it accepts a String that contains the WADL-Schema and a default package name. Then it creates a set of Java classes mapping WADL resources and depending data files.
\lstset{basicstyle={\footnotesize \ttfamily}, language=Java, numbers=left, frame=none, tabsize=2, caption=The IWadl2Java Interface, label=IWadl2Java}
\begin{lstlisting}
package at.ac.tuwien.infosys.wadl2java;

public interface IWadl2Java {
	List<JavaSource> generate(String wadlSchema, String packageName) 
		throws WadlException;
}
\end{lstlisting}
Additionally, there is an interface to access a lower-level parsing routine which maps WADL-XML files onto a Java object tree. Usually it is not required to access this interface, but in case programatic manipulation of the XML-tree is necessary, this can be quite helpful.
\lstset{basicstyle={\footnotesize \ttfamily}, language=Java, numbers=left, frame=none, tabsize=2, caption=The IApplicationParser Interface, label=IApplicationParser}
\begin{lstlisting}
package at.ac.tuwien.infosys.wadl2java.xml;

public interface IApplicationParser {
	IApplication parse(String wadlDescription) throws WadlException;
}
\end{lstlisting}

\subsubsection{Data Formats}
Java2Wadl2Java provides a direct Java mapping for WADL language elements in \emph{src/main/java/at/ac/tuwien/infosys/java2wadl/wadl}. 
\\ \\
More information on data formats is available as JavaDoc.