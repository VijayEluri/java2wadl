There are many different ways to enable (enterprise) application communication on small and large scales. The choice of the communication protocol and the corresponding middleware platform have a large impact on the resulting software architecture and dictate general application properties such as maintainability, performance and security. \cite{BIGREST} identifies four ways to integrate applications: relying on a \emph{shared database}, using \emph{batched file transfer}, calling \emph{remote procedures} (i.e. \emph{RPCs}) or exchanging \emph{(asynchronous) messages} over a message bus.
\\ \\
This article examines two means of communication in particular: The so called "Big" web service technology stack SOAP applies a combination of \emph{RPCs} and \emph{Message Passing}, while RESTful web services rely on \emph{RPCs} solely \cite{WEBDSGN}. Both technologies typically use the \emph{Hypertext Transfer Protocol} for information exchange, which has a tremendous influence on both protocols, due to the fact that HTTP is stateless and not primarily designed for machine-to-machine interaction. 
\\ \\
The lack of formal description for applications relying on the \emph{Hypertext Transfer Protocol} (i.e. Web Applications) led to the development of various \emph{wep-scraping} frameworks\footnote{The main purpose of scrapers is to extract information out of web sites. A typical tool is the Web::Scraper library available in the CPAN repository.} as well as natural language-based descriptions provided by web applications authors. Both methods are slow, insecure and ultimately error-prone, since simple changes in the markup of web-sites can render the web-scraper useless and also yield inconsistencies to the natural-language description in case it was not updated simultaneously. \cite{WADLWSDL}
\\ \\
Because of these problems WADL (Web Application Description Langauge) and WSDL (Web Service Description Language) emerged as the two major specifications providing a machine-processable means of formally describing web applications. While WADL is typically used to describe HTTP-based RESTful applications, WSDLs primary goal is to describe SOAP services. In version 2.0 the WSDL standard was extended with a feature-set that allows to describe RESTful applications.
\\ \\
This article is organized as follows: The first part introduces the general principles and underlying technologies of (RESTful) Web Services and its means of description in the form of the \emph{Web Application Description Language}. Eventually the REST architectural style and the WADL protocol are examined in detail and compared to the SOAP- and WSDL-protocols.
\\
The second part inquires the architecture and implementation of the \emph{Java2Wadl2Java} library and presents simplified, real-world usage-scenarios.